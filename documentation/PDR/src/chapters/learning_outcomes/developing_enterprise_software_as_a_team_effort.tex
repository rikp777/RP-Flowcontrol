

\section{Ontwikkelen van bedrijfssoftware als teaminspanning}\label{sec:ontwikkelen-van-bedrijfssoftware-als-teaminspanning}

%Leeruitkomst 1 - Ontwikkeling en implementatie van bedrijfssoftware Je ontwikkelt en implementeert bedrijfssoftware
%zowel
%individueel als in teamverband met behulp van een geschikt bedrijfsontwikkelingsplatform en toepassingsraamwerk, volgens een professioneel softwareontwikkelingsproces.
%Je ontwikkelt dergelijke software in teamverband en houdt daarbij rekening met zowel functionele als niet-functionele eisen zoals gesteld door de stakeholders en de wetgeving.
%Verdere toelichting In dit semester wordt bedrijfssoftware gedefinieerd als grootschalige gedistribueerde software gericht op een organisatie en in staat om grote aantallen gelijktijdige gebruikers en gegevensoverdrachten te verwerken.
%Typisch is dat deze belasting niet gelijkmatig over de tijd is verdeeld.
%De ontwikkeling van bedrijfssoftware vindt plaats in teamverband volgens agile scrum.
%Het te ontwikkelen systeem voldoet zowel aan de functionele als aan de niet-functionele eisen die door de belanghebbenden worden gesteld.
%Het voldoen aan de functionele en niet-functionele eisen wordt aangetoond door (geautomatiseerde) tests.
%Daarnaast zal het systeem voldoen aan de General Data Protection Regulation (GDPR)


%Wat wil ik leren
	Dit semester wil ik leren om enterprise software op te leveren in teamverband, aan een echte klant.
	Hierbij bespreken we samen met de klant zijn wensen en functionaliteiten.
	Voor deze samenwerking ga ik de Agile Scrum methodiek gebruiken.
	Daarnaast moet het systeem dat door ons als groep wordt ontwikkeld voldoen aan de GDPR-wetgeving.


%Wat moet ik doen om dit te kunnen bereiken?
	Dit leerdoel ga ik bereiken door een strakke planning en goede communicatie met het team en klant.
	Ook zullen we samen technische aspecten bespreken en doelgericht onderzoek uitvoeren om zo tot oplossingen te komen
	voor onze problemen.


%Welke middelen heb ik hiervoor nodig?
	Hierbij is het van belang een gemotiveerd team te hebben met leergierigen individuen en een klant die pro-actief
	meedenkt over wat hij wil in het eindproduct.
	Communicatie is hierbij belangrijk met het team, maar ook met de externe partijen.

	Tijdens ons groepsproject is er communicatie met verschillende teams en contact personen.
	Zo is er een groep bezig aan de front-end en is er een afstudeer student bezig met de technische kant van de front-end.


%Hoe ga ik success meetbaar maken?
	Ik ga dit leerdoel meetbaar maken door middel van een evaluatie op het einde van elke sprint.
	Hierbij noteer ik mijn sterke en zwakke punten die worden aangekaart door mijn team zodat ik deze in de toekomst kan verbeteren.
	Ook is het project en de deelproducten die wij als groep opleveren een goede indicatie van mijn bekwaamheid tot deze leeruitkomst.

	\newpage
	\bigskip



	\subsection{Ontwikkel process}
%Evaluatie hoe heb ik het bereikt?

%SPRINT 0 =============================================
	\subsubsection{Sprint 0}
	\paragraph{Evaluatie: 12-03-2021}
	Dit is de ontdekkingsfase; wat is het project, welke technieken komen erbij kijken
	en welke technieken kunnen we het beste toepassen voor ons project.
	Hoe gaan we als groep een succesvol project opleveren voor onze klant; wat zijn functionele- en non functionele eisen/wensen.
	Hierover hebben we samen met de klant gesproken in verschillende "online" bijeenkomsten.
	De uitkomsten hiervan zijn vastgelegd in de deel documenten. Deze zijn terug te vinden in het canvas dashboard.


	Hieronder is een opsomming van de uitkomsten die de professionaliteit van de sprint evalueren.
	Hierbij heb ik persoonlijk aan mijn teamleden gevraagd waar voor mij verbetering ligt bij deze sprint.
%Sterke punten
	\subparagraph{Sterke punten:}
	\begin{itemize}
		\setlength{\itemsep}{0pt}%
		\setlength{\parskip}{0pt}%
		\item Denkt diep na over een topic en brengt een goed technisch beeld over naar teamgenoten
		\item Levert een goede bijdrage aan het ontwikkelen
	\end{itemize}

%Verbeter punten
	\subparagraph{Verbeter punten:}
	\begin{itemize}
		\setlength{\itemsep}{0pt}%
		\setlength{\parskip}{0pt}%
		\item Nog n.v.t.
	\end{itemize}


%Evaluatie hoe heb ik het bereikt?

%SPRINT 1 =============================================
	\subsubsection{Sprint 1}
	\paragraph{Evaluatie: 01-04-2021}
	Dit was onze eerste sprint. Tijdens deze sprint hebben we de opdracht vastgesteld en een ontwerp gemaakt van ons project.
	Bij het maken van het ontwerp was het belangrijk ervoor te zorgen dat de app een modulaire opbouw krijgt, wat er voor zorgt dat het goed overdraagbaar is.
	Om deze modulariteit te behalen hebben we gekozen voor een microservice architectuur.
	Het is belangrijk dat het project overdraagbaar is, omdat dit project in de toekomst zal worden door ontwikkeld door externe project groepen.
	Ook is het van belang dat het ontwerp aan de "GDPR" regelgeving voldoet.
	De GDPR-regelgeving beschermt de gebruiker en zijn gegevens in ons systeem.
	Ook hebben we tijdens deze sprint verschillende prototypes gemaakt waarvan de klant zeer onder de indruk was.
	In deze prototypes hebben we nieuwe technische methodes uitwerkt en getest, die we laten willen gaan toepassen in het definitieve project.
	Deze ontwerpen en prototypes zijn ook besproken met vak experts / docenten.
	Dit stelt de kwaliteit vast van onze deelproducten en zo kunnen we vroegtijdig de nodige wijzigingen aanbrengen.
	Ook is deze documentatie belangrijk voor de overdraagbaarheid aan een toekomstig team die aan dit project verder
	zullen gaan werken.


	Hieronder is een opsomming te vinden van de eindbespreking van sprint 1.
	Hierbij hebben we net zoals voorgaande sprint de tips en tops besproken en waar voor mij verbeteringen liggen in de
	sprints die nog komen.
%Sterke punten
	\subparagraph{Sterke punten:}
	\begin{itemize}
		\setlength{\itemsep}{0pt}%
		\setlength{\parskip}{0pt}%
		\item Initiatief nemend.
		\item Gemotiveerd en leergierig.
		\item Brede technische kennis.
		\item Je zoekt dingen tot het bot uit.
		\item Algemene samenwerking; je helpt waarnodig mee en helpt anderen goed.
	\end{itemize}

%Verbeter punten
	\subparagraph{Verbeter punten:}
	\begin{itemize}
		\setlength{\itemsep}{0pt}%
		\setlength{\parskip}{0pt}%
		\item Te lang zichzelf vast kunnen klampen aan een ding.
		Perfectie is een streven maar het moet niet tegen werken.
		\item Je kan meer initiatief tonen om aan nieuwe zaken te beginnen wanneer je huidige af zijn.
	\end{itemize}

	\bigskip

%SPRINT 2 =============================================
	\subsubsection{Sprint 2}
	\paragraph{Evaluatie: 23-04-2021}
	Dit was onze eerste echte print oplevering waar we ook echt code gingen opleveren.
	Hierbij werken we volgens de agile scrum methodiek en hebben we elke 2 weken een sprint oplevering.
	Deze sprint oplevering presenteren we ons werk voor de verschillende teams, de project eigenaar Wouter en
	natuurlijk bespreken we naderhand hoe de oplevering is gegaan met onze docenten.
	Door de positieve feedback en de goede samenwerking en teaminspanning bij het project kan ik mijzelf
	beoordelen met een "proficient".
	Natuurlijk zijn we nog niet op het einde van het semester daarom zie ik deze proficient onder voorbehoud.
	In de aankomende sprint ga ik deze beoordeling definitief maken.





%Eigen beoordeling
\subsection{Eind beoordeling / reflectie}
%Wij als groep hebben een mooi degelijk product opgeleverd.
%In dit project zijn alle wensen verwerkt die door de klant werden opgedragen.
%Ook hebben we in dit project alle technieken die voor dit semester van toepassing waren verwerkt.
%Dit project kan in de toekomst verder worden doorontwikkeld door andere project groepen.
%
%\bigskip
%Door de behaalde resulaten difinieer ik mijn bekwaamheid tot dit leerdoel als:\\
%\par\vspace{10pt}\textbf{\uppercase{"Proficient"}}\\
	Binnen ons team is er een ontzettend goede communicatie. Afspraken waren daarom ook duidelijk binnen het gehele team.
	Ook ondersteunde we elkaar waar nodig, sommigen meer dan anderen maar over het algemeen ben ik zeer tevreden.

\newpage
