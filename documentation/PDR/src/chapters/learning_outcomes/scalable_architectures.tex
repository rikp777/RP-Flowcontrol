%! Author = rikpe
%! Date = 24/04/2021

\section{Schaalbare architecturen}\label{sec:schaalbare-architecturen}
%Leer uitkomst 4 - Schaalbare architecturen Je ontwikkelt bedrijfssoftware op basis van een gekozen gedistribueerde
%architectuur die duidelijk schaalbaarheid ondersteunt voor hoog volume communicatie en event handling, en
%onafhankelijk life cycle management mogelijk maakt.
%Verdere toelichting Tegenwoordig zijn schaalbare architecturen vaak gebaseerd op microservices.
%Microservices maken fijnkorrelige services in een gedistribueerd systeem mogelijk, waarbij elke service zijn eigen
%levenscyclus heeft.
%Je definieert interfaces tussen microservices met behulp van event storming, wat een workshop-gebaseerde methode is
%om alle relevante gebeurtenissen te ontdekken die binnen het domein van het bedrijfssysteem plaatsvinden.
%Je implementeert samenwerking tussen services met behulp van event sourcing technieken.
%Je documenteert je architectuurontwerp met diagrammen volgens een ontwerpproces.


%Wat wil ik leren
Ik wil leren hoe ik enterprise software maak die schaalbaar en gedistribueerd is.
Binnen de enterprise software moeten de microservices onafhankelijk van elkaar opereren.
Microservices zijn onafhankelijke programma's die met elkaar kunnen communiceren door middel van protocollen.
Dit is mogelijk door een goed technische ontwerp van het op te leveren product.
Dit ontwerp wordt gedocumenteerd in verschillende technische documenten zoals het Software Architecture Document (SAD).

%Wat moet ik doen om dit te kunnen bereiken?
Hierbij is het van belang dat ik de nodige kennis op doe over microservices.
Daarnaast zal ik onderzoek moeten doen naar hoe deze protocollen onderling werken en hoe ik deze moet toepassen.

%Welke middelen heb ik hiervoor nodig?
Hierbij is het ook van belang om goede tooling toe te passen zoals UML of C4 voor een goed technische ontwerp.
Bij het ontwikkelen is het ook belangrijk om een goede en intelligente IDE te gebruiken, zodat ik efficient te werk
kan gaan.

%Hoe ga ik success meetbaar maken?
Ik ga dit leerdoel meetbaar maken door een goed project op te leveren zowel voor het individueel project als het
groeps project.
Hierbij is het belangrijk dat ik stevige argumenten heb om mijn keuzes te onderbouwen die ik heb gemaakt tijdens
het ontwikkel process.
Het project moet natuurlijk schaalbaar zijn dit ga ik aantonen door een microservice architectuur toe te passen.

Dit leerdoel sluit ook goed aan bij het leerdoel "Voorbereiding op levenslang leren" in dit hoofdstuk omschrijf ik
hoe ik microservices heb toegepast voor mijn individueel project.
Graag wil ik u de lezer dan ook verwijzen om meer te lezen over microservices in dat hoofdstuk.












\newpage
