%! Author = rikpe
%! Date = 24/04/2021

\section{Context Based Research}\label{sec:context-based-research}

%Leeruitkomst 2 - Contextgericht onderzoek Je onderbouwt je keuzes voor processen en technieken aan de hand van een
%algemeen geaccepteerde onderzoeksmethode en rekening houdend met je eigen ethische waarden.
%Verdere toelichting Je onderbouwt je keuzes voor processen en technieken aan de hand van contextgebaseerd onderzoek.
%Je maakt gebruik van bekende en algemeen geaccepteerde onderzoeksmethoden of -modellen, zoals het DOT Framework en
%ICT Research Methods.
%Je communiceert je onderzoeksaanpak, plan, resultaten en conclusie zowel mondeling als schriftelijk.

%Wat wil ik leren
Dit semester wil ik leren om doelgericht onderzoek te doen dat kan bijdragen een stevig
standpunt te creëren bij het ontwikkelen van technische en complexe software oplossingen.
Onderzoek is nodig voor kennis bij een bepaald onderwerp en om deze zo beter te kunnen begrijpen.
Vanuit daar kun je je kennis stapsgewijs (onderzoek voor onderzoek) opbouwen om zo tot een gepaste oplossing te
komen voor een probleem.


%Wat moet ik doen om dit te kunnen bereiken? / %Welke middelen heb ik hiervoor nodig?
Door de onderzoeksmethodieken te gebruiken die in het DOT-framework staan omschreven kan ik doelgericht onderzoek
doen.
Voor het onderzoek ga ik verschillende methodieken combineren om zo tot een oplossing te komen op mijn probleem.


%Hoe ga ik success meetbaar maken?
Dit leerdoel ga ik meetbaar maken door onderzoeksdocumenten op te leveren die door zowel mij als door een ander als
nuttig worden beschouwd en daarmee ook een bijdrage kunnen leveren bij een oplossing tot een complex probleem.
Ik definieer mijn kwaliteit op de feedback die ik heb ontvangen van vak bekwame experts "docenten".

%Wat heb ik gedaan om dit aan te tonen?
Bij zowel het groepsproject als het individueel project heb ik verschillende casestudies gemaakt die voor een klant
kunnen helpen bij het maken van een keuze om zo een probleem op te lossen.

\bigskip
\subsection{Ontwikkel process}
%evaluatie / %Beoordeling
\subsubsection{Case Studies}
\paragraph{Evaluatie: 24-04-2021}
Tijdens het groepsproject hebben we zes verschillende case studies moeten uitwerken.
Deze case studies gaven een complex probleem waar een oplossing bij gevonden moest worden.
De problemen deden zich voor op verschillende aspecten binnen het ontwikkelen van software.
De case studies zijn uitgewerkt door middel van het toepassen van verschillende methodieken uit het DOT-framework.
De casestudies die tijdens de proftaak groep zijn behandeld zijn goed afgerond en de feedback van de leraren is
verwerkt, daarom beschouw ik mijn huidige oriëntatie als "georiënteerd".
Deze case studies zijn terug te vinden in het canvas overzicht opdrachten overzicht van project groep '3'.
Bij de verschillende case studies hebben we taken verdeeld mijn bijdragen aan de case studies was als volgt:

\begin{itemize}
	\setlength{\itemsep}{0pt}%
	\setlength{\parskip}{0pt}%
	\item Gebrainstormed over de case studie aan zich wat houdt deze in en wat vragen ze van ons
	\item Onderzoeks vragen opstellen gebaseerd op de brainstorm sessie
	\item Onderbouwde feedback gevens op elkaars uitwerking op een vraag en waar nodig ondersteunen.
	\item Onderzoeks vragen en deelvragen uitwerken.

\end{itemize}

Gebaseerd op de feedback die we terug krijgen gekoppeld van onze docenten ligt onze kwaliteit hoog.
Alle feedback die we ontvangen wordt genotuleerd en wordt in de volgende case studie meegebracht.

\subsubsection{Onderzoeken project}
\paragraph{Evaluatie: 05-04-2021}
Voor het groepproject zijn een twee onderzoeken uitgevoerd.

\subsection{Eind beoordeling / reflectie}
%Gebaseerd op de verschillende case studies en groeps onderzoeken beschouw ik mijn huidige bekwaamheid tot dit leerdoel op:\\
%\par\vspace{10pt}\textbf{\uppercase{"Proficient"}}\\
\newpage
