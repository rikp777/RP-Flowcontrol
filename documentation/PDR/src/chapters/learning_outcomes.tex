\chapter{Leeruitkomsten}
\label{ch:learning_outcomes}


%an introduction stating the learning outcome. (~50 words)
%a self-evaluation stating at what level of progression (see above), you believe you currently are for this learning
%outcome.
%a learning process description that supports your self-evaluation (why do you believe you have reached the
%above-mentioned level?), and the received feedback from the teachers and fellow students.
%Refrain from adding full pieces of work that can also be found in Canvas.
%Add a personal reflection on the received feedback: what do you think of it, and what did you do with it?






\section{Ontwikkelen van bedrijfssoftware als teaminspanning}\label{sec:ontwikkelen-van-bedrijfssoftware-als-teaminspanning}

%Leeruitkomst 1 - Ontwikkeling en implementatie van bedrijfssoftware Je ontwikkelt en implementeert bedrijfssoftware
%zowel
%individueel als in teamverband met behulp van een geschikt bedrijfsontwikkelingsplatform en toepassingsraamwerk, volgens een professioneel softwareontwikkelingsproces.
%Je ontwikkelt dergelijke software in teamverband en houdt daarbij rekening met zowel functionele als niet-functionele eisen zoals gesteld door de stakeholders en de wetgeving.
%Verdere toelichting In dit semester wordt bedrijfssoftware gedefinieerd als grootschalige gedistribueerde software gericht op een organisatie en in staat om grote aantallen gelijktijdige gebruikers en gegevensoverdrachten te verwerken.
%Typisch is dat deze belasting niet gelijkmatig over de tijd is verdeeld.
%De ontwikkeling van bedrijfssoftware vindt plaats in teamverband volgens agile scrum.
%Het te ontwikkelen systeem voldoet zowel aan de functionele als aan de niet-functionele eisen die door de belanghebbenden worden gesteld.
%Het voldoen aan de functionele en niet-functionele eisen wordt aangetoond door (geautomatiseerde) tests.
%Daarnaast zal het systeem voldoen aan de General Data Protection Regulation (GDPR)


%Wat wil ik leren
Dit semester wil ik leren om enterprise software op te leveren in teamverband, aan een echte klant.
Hierbij bespreken we samen met de klant zijn wensen en functionaliteiten.
Voor deze samenwerking ga ik de agile scrum methodiek gebruiken.
Daarnaast moet het systeem dat door ons als groep wordt ontwikkeld voldoen aan de GDPR-wetgeving.


%Wat moet ik doen om dit te kunnen bereiken?
Dit leerdoel ga ik bereiken door een strakke planning en goede communicatie met het team en klant.
Ook zullen we samen technische aspecten bespreken en doelgericht onderzoek uitvoeren om zo tot oplossingen te komen
voor onze problemen.


%Welke middelen heb ik hiervoor nodig?
Hierbij is het van belang een gemotiveerd team te hebben met leergierigen individuen en een klant die pro-actief
meedenkt over wat hij wilt in het eindproduct.


%Hoe ga ik success meetbaar maken?
Ik ga dit leerdoel meetbaar maken door middel van een evaluatie op het einde van elke sprint. Hierbij noteer ik
mijn sterke en zwakke punten die worden aangekaart door mijn team zodat ik deze in de toekomst kan verbeteren.

\bigskip
%Evaluatie hoe heb ik het bereikt?
Evaluatie: 12-03-2021

%Sterke punten
Denkt diep na over een topic en brengt een goed technisch beeld over naar teamgenoten.
Levert een goede bijdragen aan het ontwikkelen.

%Verbeter punten
%n.v.t

\bigskip
%Eigen beoordeling
Ik ben zeer tevreden met mijn eerste evaluatie.


\section{Context Based Research}\label{sec:context-based-research}

%Leeruitkomst 2 - Contextgericht onderzoek Je onderbouwt je keuzes voor processen en technieken aan de hand van een
%algemeen geaccepteerde onderzoeksmethode en rekening houdend met je eigen ethische waarden.
%Verdere toelichting Je onderbouwt je keuzes voor processen en technieken aan de hand van contextgebaseerd onderzoek.
%Je maakt gebruik van bekende en algemeen geaccepteerde onderzoeksmethoden of -modellen, zoals het DOT Framework en
%ICT Research Methods.
%Je communiceert je onderzoeksaanpak, plan, resultaten en conclusie zowel mondeling als schriftelijk.

%Wat wil ik leren
Dit semester wil ik leren om doelgericht onderzoek te doen die kan bijdragen een stevig
standpunt te creëren bij het ontwikkelen van technische en complexe software oplossingen.
Onderzoek is nodig voor kennis bij een bepaald onderwerp en om deze zo beter te kunnen begrijpen.
Van uit daar kun je je kennis stapsgewijs (onderzoek voor onderzoek) opbouwen om zo tot een gepaste oplossing te
komen voor een probleem.


%Wat moet ik doen om dit te kunnen bereiken? / %Welke middelen heb ik hiervoor nodig?
Door de onderzoeksmethodieken te gebruiken die in het DOT-framework staan omschreven kan ik doelgericht onderzoek
doen.
Voor het onderzoek ga ik verschillende methodieken combineren om zo tot een oplossing te komen op mijn probleem.


%Hoe ga ik success meetbaar maken?
Dit leerdoel ga ik meetbaar maken door onderzoeksdocumenten op te leveren die door zowel mij als door een ander als
nuttig worden beschouwd en daarmee ook een bijdragen kunnen leveren bij een oplossing tot een complex probleem.

%Wat heb ik gedaan om tdit aan te tonen
Bij zowel het groepsproject als het individueel project heb ik verschillende case studies gemaakt die voor een klant
kunnen helpen bij het maken van een keuze om zo een probleem op te lossen.

\bigskip
%evaluatie / %Beoordeling
Evaluatie: 13-03-2021

Deze case studies zijn goed afgerond en de feedback van de leraren is verwerkt, daarom beschouw ik mijn huidige
oriëntatie als "georiënteerd".




\section{Voorbereiding op levenslang leren}\label{sec:voorbereiding-op-levenslang-leren}
%Leer uitkomst 3 - Voorbereiding op levenslang leren Je signaleert opkomende trends in software engineering,
%onderzoekt ze en past ze waar nodig toe in je projecten.
%Ook ben je je bewust van mogelijke carrièrepaden en je verwerft de vereiste vaardigheden om voorbereid te zijn op je
%toekomstige carrière.
%Verdere toelichting Niet alle opkomende trends komen aan bod in de andere eindtermen.
%Opkomende trends zijn ook Domain-Driven Design, Blockchain, Programmeerparadigma's, Machine Learning, en Quantum
%Computing.
%De genoemde onderwerpen komen in het studiemateriaal aan bod.
%Je kiest één van deze onderwerpen in overleg met je tutorgroep en docenten.
%Carrièrepaden zullen per student verschillen.
%Software engineers kunnen bijvoorbeeld doorgroeien naar software architect of teamleider.
%Studenten die zich specialiseren in bijvoorbeeld cybersecurity, toegepaste data science of game design kunnen een
%ander carri.èrepad hebben dan studenten die zich specialiseren in software engineering.
%Je moet je bewust zijn van je eigen carrièrepad en de vaardigheden die voor die rol vereist zijn.
%Je moet bereid zijn om de volgende stap te zetten om je vaardigheden te ontwikkelen, wat een minor of een
%afstudeerproject kan betekenen.

%Wat wil ik leren
Voor mijn studie worden veel nieuwe technieken aangedragen door school, maar hoe ga ik er in de toekomst zelf voor
zorgen dat ik up-to-date blijf met de laatste trends die zich in de ICT-wereld ontwikkelen.
Hiervoor is het daarom belangrijk nu al na te denken over hoe ik dit ga aanpakken.
Daarom wil ik dit semester leren om mijzelf voor te bereiden om levenslang te leren.

%Wat moet ik doen om dit te kunnen bereiken?
Om dit te kunnen bereiken moet ik mij zelf blijven motiveren om door te leren na mijn studie.



\subsection{Deployment en schalen}\label{subsec:deployen-en-schalen}

	\subsubsection{Eerste evaluatie - Week 3}\label{subsubsec:eerste-evaluatie---week-3}
	De eerst weken heb ik onderzoek gedaan hoe docker werkt en hoe ik een image kan maken van een project.
	Vervolgens heb ik een compose file gemaakt die het mogelijk maakt verschillende images gelijk matig te starten.

   \subsection{Protocollen}\label{subsec:protocollen}
	\subsubsection{Eerste evaluatie - Week 3}
	De eerst weken heb ik onderzoek gedaan naar de verschillende protocollen die worden gebruikt voor het streamen
	van data.
	Welke protocollen passen bijvoorbeeld het beste bij het huidige groepsproject.
	UTP of TCP / rtmp of iets anders.


\section{Schaalbare architecturen}\label{sec:schaalbare-architecturen}
Leer uitkomst 4 - Schaalbare architecturen Je ontwikkelt bedrijfssoftware op basis van een gekozen gedistribueerde
architectuur die duidelijk schaalbaarheid ondersteunt voor hoog volume communicatie en event handling, en onafhankelijk life cycle management mogelijk maakt.
Verdere toelichting Tegenwoordig zijn schaalbare architecturen vaak gebaseerd op microservices.
Microservices maken fijnkorrelige services in een gedistribueerd systeem mogelijk, waarbij elke service zijn eigen levenscyclus heeft.
Je definieert interfaces tussen microservices met behulp van event storming, wat een workshop-gebaseerde methode is om alle relevante gebeurtenissen te ontdekken die binnen het domein van het bedrijfssysteem plaatsvinden.
Je implementeert samenwerking tussen services met behulp van event sourcing technieken.
Je documenteert je architectuurontwerp met diagrammen volgens een ontwerpproces.






\section{Ontwikkeling en uitvoering (DevOps)}\label{sec:ontwikkeling-en-uitvoering-(devops)}
Leer uitkomst 5 - Development and Operations (DevOps) Je zet een omgeving en teamprocessen op die een volledig geautomatiseerde softwarelevenscyclus ondersteunen, terwijl je een hoge kwaliteit, hoge beschikbaarheid, snelle levering en korte releasetijden garandeert.
Verdere verduidelijking Wijzigingen in de software worden ingevoerd in een goed gedefinieerd proces volgens change management en release procedures.
U ontwikkelt bedrijfssoftware met behulp van CI/CD-pijplijnen.
CI/CD staat voor continuous integration en continuous delivery en/of continuous deployment.
De CI/CD-pijplijn voert geautomatiseerde tests uit, waaronder unit tests, component tests, integratietests en gebruikersacceptatietests.
Daarnaast wordt een rapport over code coverage en statische code kwaliteit gegenereerd.
De CI/CD pipeline, testomgeving en deployment worden gecontaineriseerd.
De implementatie van (micro)services zal worden opgeschaald wanneer nodig met behulp van container orchestration.






\section{Clouddiensten}\label{sec:clouddiensten}
Leer uitkomst 6 - Cloud Services Je integreert cloud services en serverless computing technieken die je enterprise applicatie ondersteunen en er goed bij passen.
Je onderzoekt de kosten en hoeveelheid resources die nodig zijn voor je applicatie.
Je keuzes voor cloudprovider en ondersteunde tooling zijn gebaseerd op de belangen van stakeholders.
Verdere toelichting Mogelijke clouddiensten zijn schaalbare databases, authenticatie, logging, monitoring, virtuele machines, en containerisatie.
Het is ook mogelijk om een (micro)service in te zetten door alleen broncode aan te leveren, dit wordt serverless computing of function as a service (FAAS) genoemd.
Je kent de voor- en nadelen van FAAS ten opzichte van containerisatie en je bent je bewust van de kosten die gepaard gaan met clouddiensten.






\section{Veiligheid door ontwerp}\label{sec:veiligheid-door-ontwerp}
Leer uitkomst 7 - Security by Design Je integreert authenticatie en autorisatie, en beperkt mogelijke beveiligingsinbreuken tijdens het ontwerp en de implementatie van bedrijfssoftware.
Verdere toelichting Je onderzoekt mogelijke beveiligingsinbreuken door misbruikgevallen te definiëren en je houdt rekening met deze gevallen tijdens het ontwerp en de implementatie.
Je implementeert authenticatie en autorisatie met behulp van tooling en libraries die door enterprise software frameworks worden geleverd.
Je bent op de hoogte van de meest kritische beveiligingsrisico's en je zorgt ervoor dat deze risico's voor jouw
applicatie worden geminimaliseerd door de OWASP top 10 beveiligingsrisico's van webapplicaties aan te pakken.







\section{Gedistribueerde gegevens}\label{sec:gedistribueerde-gegevens}
Leer uitkomst 8 - Gedistribueerde data Je bent je bewust van data-eisen en je ontwikkelt bedrijfssystemen die gebruik maken van gedistribueerde data tooling en best practices.
Je hebt een kritische houding ten opzichte van mogelijke privacy en ethische kwesties.
Verdere toelichting Je definieert data-eisen, gebaseerd op de context van de data, de architectuur van het systeem, de infrastructuur, en (niet-)functionele eisen.
Je bent op de hoogte van de beschikbaarheid van verschillende soorten gedistribueerde data-tooling en hun onderscheidende kenmerken en verschillende toepassingsgebieden.
Je integreert gedistribueerde data tooling en best practices in de architectuur van je systeem en onderbouwt waarom deze beslissingen bijdragen aan het voldoen aan data-eisen en niet-functionele eisen (bijv.
schaalbaarheid, performance, security, en wetgeving waaronder GDPR).
