\chapter{Inleiding}
\label{ch:introduction}
(~500 words)

Dit PDR-verslag documenteert de verschillende processen die hebben plaatsgevonden voor het ontwikkelen van het
uiteindelijke product.
In dit geval is dat de software 'Flowcontrol' die het distributieproces voor distributiecentra digitaliseert en
automatiseert.
Ook het groepsproject 'Smilie' komt aan bod.
Verder worden de knelpunten, leermomenten en onderzoeksvragen nauwkeurig behandeld.
Tot slot worden de conclusies en aanbevelingen voor het project beschreven.


\section*{Individueel}
Voor mijn individueel project ga ik een ERP-pakket ontwikkelen genaamd Flowcontrol.
Flowcontrol gaat er voor zorgen dat verschillende processen die binnen een discributiecentrum worden uitgevoerd,
worden gedigitaliseerd maar ook worden geautomatiseerd.
Hierdoor hebben medewerkers meer tijd voor andere werkzaamheden.

\section*{Groep}
Sinds covid-19 maakt de GGZ gebruik van videoverbindingen om hun cliënten te bellen om contactmomenten te voorkomen.
Een nadeel hiervan is dat het voor medewerkers soms moeilijk is om in te schatten hoe cliënten zich voelen.
Hiervoor is de oplossing bedacht om beter op te merken hoe cliënten zich voelen, door middel van het meten van de
hartslag en
galvanische huidreacties.
In de toekomst zullen ook gezichtsuitdrukkingen en de toon van de stem worden gemeten.
De resultaten van deze metingen zijn gevoelige informatie en moeten veilig worden verstuurd naar de diensten die deze gegevens omzetten in logische informatie.

Onze groep gaat werken aan de veilige overgang van gegevens.
Onze groep zal er ook voor zorgen dat de diensten in de cloud zullen gaan draaien in plaats van lokaal, wat nu gebeurt.
