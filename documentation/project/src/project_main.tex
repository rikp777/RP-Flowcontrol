%& -job-name=Project_Rik_Peeters

\documentclass[11pt, twoside]{report}

\usepackage{packages}
\usepackage[utf8]{inputenc}
\usepackage[T1]{fontenc}

%! Author = rikpe
%! Date = 02/03/2021

% Document
\begin{document}

    \begin{titlepage}
    \begin{center}
        \vspace*{0.1cm}

        \Huge
        \textbf{\documentTitle}

        \vspace{0.3cm}
        \LARGE
        \documentSubtitle\\

        \vspace*{1.5cm}
        \includegraphics[width=0.6\textwidth]{images/books.jpg}

        \vspace*{3.5cm}
        \begin{minipage}{0.5\textwidth}
            \centering
            \fcolorbox{purple}{purple}{\includegraphics[width=0.5\linewidth]{images/profile_image.jpg}}
        \end{minipage}%
        \begin{minipage}{0.5\textwidth}
            \centering
            \includegraphics[width=0.9\textwidth]{images/fontys_logo.png}
        \end{minipage}%
    \end{center}
    \vfill

    \normalsize
    \begin{minipage}{1\textwidth}
        \textit{Autheur: \textbf{\documentAuthor \newline}}
        \textit{Datum: \textbf{\documentDate \newline}}
        \textit{Versie: \textbf{\documentVersion \newline}}
    \end{minipage}%

    \vspace*{1.5cm}

    \begin{center}
        \textit{\documentAuthorBio}
    \end{center}
\end{titlepage}

    \newpage
    \tableofcontents
    \newpage

    \clearpage
    \setcounter{page}{1}

    \chapter{Inleiding}
    \label{ch:inleiding}
    Dit verslag documenteert het project Flowcontrol en de technische aspecten hiervan.
    Het project dat ik ga ontwikkelen gaat ervoor zorgen dat data wordt gedigitaliseerd en dat processen worden
    geautomatiseerd.
    Hierdoor zullen werknemer meer tijd overhouden voor andere werkzaamheden die van belang zijn.
    Omdat ik zelf binnen een distributiecentrum werkzaam ben weet ik van het reilen en zeilen ben ik mij bewust van
    het handmatig doorvoeren van de orders.
    Het process is zeer omslachtig en dit kan natuurlijk verbeterd worden.
    Vandaar mijn motivatie om dit project te
    ontwikkelen.
    Tevredenheid bij klanten en organisaties is altijd belangrijk voor ieder bedrijf.
    Daarom wil ik ervoor zorgen dat het product dat ik ga ontwikkelen een bijdrage hieraan kan bieden.

    \chapter{Project}
    \label{ch:project}
    \section{Context}
    \label{sec:context}
    Voor mijn individueel project ga ik een distributie app ontwikkelen genaamd Flowcontrol.
    Deze app gaat er voor zorgen dat het distributieproces gedigitaliseerd wordt maar ook geautomatiseerd.
    De processen die worden gedigitaliseerd zijn onder te verdelen in verschillende modules waardoor het project zeer modulair wordt
    Deze modules bestaan bijvoorbeeld uit: een aanlever module (kweker), een planning module, een productie module,
    een transport module en een distributie module en ga zo maar door.

    \section{Doel van het project}
    \label{sec:doel-van-het-project}
    Het doel van dit project is dat de oude excel-sheets en papieren handel in de prullenbak kan worden gegooid.
    Het distributie proces wordt hierdoor simpelweg sneller en makkelijker.
    Het doel is ook dat het project inzicht geeft in bepaalde zaken die vroeger nooit overzichtelijk waren.
    Hierbij kun je bijvoorbeeld denken aan precieze tijd registratie, wanneer wat wordt afgehandeld.

    \newline

    Het bovenstaande project wordt Flowcontrol genoemd. Ik kies ervoor dit project op te splitsen in verschillende
    deeltrajecten genoemd modules. Mijn focus ligt op de module productie.

    \newpage

    \chapter{Sprints}
    \label{ch:sprints}

    \section{Github links}\label{sec:github-links}
    Github links: \\
    \newline
    Documentatie:
    \begin{itemize}
        \item \href{https://github.com/rikp777/RP-Flowcontrol/blob/master/documentation/PDR/out/PDR_Rik_Peeters.pdf}{PDR}
        \item \href{https://github.com/rikp777/RP-Flowcontrol/blob/master/documentation/project/out/Project_Rik_Peeters.pdf}{Project}
        \item \href{https://github.com/rikp777/RP-Flowcontrol/blob/master/documentation/context_based_research/out/context_based_research_Rik_Peeters.pdf}{Context based research}
        \item \href{https://github.com/rikp777/RP-Flowcontrol/blob/master/documentation/SWOT/out/SWOT_Rik_Peeters.pdf}{SWOT}
    \end{itemize}
    \newline
    Tools:
    \begin{itemize}
        \item \href{https://github.com/rikp777/RP-Flowcontrol/issues}{Github backlog}
        \item \href{https://github.com/rikp777/RP-Flowcontrol/projects/1}{Github kanban bord}
        \item Github actions CI/CD n.v.t.
    \end{itemize}

    \newpage
    \section{Sprint 0}
    \label{sec:sprint-0}

    \subsection{Goals}\label{subsec:goals-0}
    In de eerste sprint heb ik mijn focus gelegd op het bedenken van mijn project idee "Flowcontrol".
    Dit idee is tot stand gekomen aan de hand van mijn eigen ervaringen binnen deze branch.

    \subsection{Achievements}\label{subsec:achievements-0}
    Tijdens deze sprint heb ik mijn hoofd onderzoeksvraag bedacht.
    Deze zal ik in de aankomende sprints verder uitwerken

    \subsection{Project management tool}\label{subsec:project-management-tool}
    Als project management tool heb ik ervoor gekozen om de omgeving van Github te gebruiken met issue bord als
    backlog en het kanban project bord.
    Ik heb hiervoor gekozen omdat Github vrij nieuw is en ik graag wil ervaren hoe dit werkt.
    Ook mijn CI/CD module zal in de omgeving Github worden ondergebracht doormiddel van Github actions.

    \subsection{Retrospective}\label{subsec:retrospective-0}
    In de eerste week heb ik veel tijd besteed aan het uitdenken van het idee.
    Ik ben zeer tevreden met hoe ik dit idee heb uitgedacht en de modulariteit hiervan.

    \subsection{Volgende sprint}\label{subsec:volgende-sprint-0}
    In de volgende sprint ga ik mijn onderzoeksvraag uitwerken.
    Hierbij ga ik verschillende prototypes maken.
    Zo heb ik meer inzicht op wat ik ga ontwikkelen en vooral hoe.

    \newpage
    \section{Sprint 1}
    \label{sec:sprint-1}

    \subsection{Goals}\label{subsec:goals=1}
    Tijdens sprint 1 heb ik mijn prototype gemaakt en getoond aan verschillende docenten.
    Hierbij heb ik feedback gekregen die ik in toekomstige sprints ga verwerken.
    In het prototype heb ik de communicatie tussen de verschillende microservices gebaseerd op het https
    -protocol.
    In latere sprint ga ik kijken of hier een betere oplossing voor is.
    Ook is in dit prototype het workflow algorithm verwerkt.
    Dit algoritme controleert wanneer iets wel en niet mag gebeuren en baseert hier vervolgens de actie op die wordt
    uitgevoerd.


    \subsection{Achievements}\label{subsec:achievements-1}
    Tijdens deze sprint heb ik met verschillende nieuwe technologieën geëxperimenteerd zoals HATEOAS HAl en Swagger.
    Ook heb ik gekeken hoe de logging werkt in springboot met backlog en heb hierbij de koppeling gemaakt naar logz.io.

    \subsection{Retrospective}\label{subsec:retrospective-1}
    Ik ben zeer tevreden met de prototype die ik heb gemaakt gebaseerd op het ontwerp.
    Deze ontwerpen zijn te vinden in de map genaamd 'assets'.

    \subsection{Volgende sprint}\label{subsec:volgende-sprint-1}
    In de volgende sprint wil ik mij gaan focussen op de gateway module die de applicatie moet gaan beveiligen en er
    voor gaan zorgen dat requests naar de juiste service worden gelinkt.
\end{document}
