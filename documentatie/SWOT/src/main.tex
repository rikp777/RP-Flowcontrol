%& -job-name=SWOT_Rik_Peeters

\documentclass[11pt, twoside]{report}

\usepackage{packages}

%! Author = rikpe
%! Date = 01/03/2021

% Document
\begin{document}

    \begin{titlepage}
    \begin{center}
        \vspace*{0.1cm}

        \Huge
        \textbf{\documentTitle}

        \vspace{0.3cm}
        \LARGE
        \documentSubtitle\\

        \vspace*{1.5cm}
        \includegraphics[width=0.6\textwidth]{images/books.jpg}

        \vspace*{3.5cm}
        \begin{minipage}{0.5\textwidth}
            \centering
            \fcolorbox{purple}{purple}{\includegraphics[width=0.5\linewidth]{images/profile_image.jpg}}
        \end{minipage}%
        \begin{minipage}{0.5\textwidth}
            \centering
            \includegraphics[width=0.9\textwidth]{images/fontys_logo.png}
        \end{minipage}%
    \end{center}
    \vfill

    \normalsize
    \begin{minipage}{1\textwidth}
        \textit{Autheur: \textbf{\documentAuthor \newline}}
        \textit{Datum: \textbf{\documentDate \newline}}
        \textit{Versie: \textbf{\documentVersion \newline}}
    \end{minipage}%

    \vspace*{1.5cm}

    \begin{center}
        \textit{\documentAuthorBio}
    \end{center}
\end{titlepage}

    \newpage
    \tableofcontents
    \newpage

    \clearpage
    \setcounter{page}{1}

    \chapter{Inleiding}
    \label{ch:inleiding}

    \begin{flushleft}
        Dit document bevat de SWOT-analyse die ik heb verricht aan de hand van mijn individueel- en proftaak project.
        Hierin vertel ik over mijn zwakke maar ook sterke kanten en bespreken we de kansen/bedreigen die deze kanten kunnen vormen omtrent de projecten.
    \end{flushleft}
    \begin{flushleft}
        \fbox{\begin{minipage}{0.95\textwidth}
                  \centering
                  SWOT
                  \begin{flushright}
                      \colorbox{Kansen}{\tiny{kansen}}
                      \colorbox{Bedreigingen}{\tiny{bedreigingen}}
                      \colorbox{Sterktes}{\tiny{sterktes}}
                      \colorbox{Zwaktes}{\tiny{zwaktes}}
                  \end{flushright}
        \end{minipage}}\newline
    \end{flushleft}
    \newpage


    \chapter{SWOT-analyse}\label{ch:swot-analyse}
    Een SWOT-analyse is een visueel hulpmiddel dat kan worden gebruikt om specifieke sterktes en zwaktes te identificeren in zakelijke en persoonlijke situaties. Het helpt met besluitvorming en vooruitplannen.
    Een SWOT-diagram wordt gevormd door een raster van twee bij twee.
    Elk kwadrant bevat een beschrijving van de sterktes, zwaktes, kansen en bedreigingen van het subject (SWOT staat voor strengths, weaknesses, opportunities en threats). In dit geval ben ik het subject.

    \section{Sterktes}\label{sec:sterktes}

    In dit hoofdstuk ga ik vertellen wat mijn sterke kanten zijn.

    Op het gebied van persoonlijke ontwikkeling:
    \begin{itemize}
        \item[] Ik ben zeer leergierig aangelegd en wil bij complexe problemen altijd de beste oplossing zoeken.
        \item[] Ik ben een doorzetter, ik wil hoe dan ook dat een project slaagt.
        \item[]
    \end{itemize}

    Op gebied van het project:
    \begin{itemize}
        \item[] Flowcontrol is een uitdagend en groot, modulair project met enorm veel ruimte voor uitbreiding.
        \item[] Het project is zeer modulair en heeft veel verschillende modules. Hierdoor kan ik mij verder blijven ontwikkelen.
        \item[] Het project biedt de mogelijkheid elk leerdoel aan te raken.
    \end{itemize}

    Op het gebied van teamverband:
    \begin{itemize}
        \item[] Ik kan goed werken in teamverband. Binnen deze samenwerking streef ik altijd naar een duidelijke communicatie. Ook trek ik aan de bel als zaken onduidelijk zijn zodat we deze kunnen ophelderen.
        \item[] Het ontwikkelen van software in teamverband vind ik erg leuk. Ook vind ik het belangrijk te leren van elkaars fouten en positieve punten.
    \end{itemize}

    \section{Zwaktes}\label{sec:zwaktes}
    In dit hoofdstuk ga ik vertellen wat mijn zwakkere kanten zijn.

    Op persoonlijk gebied:
    \begin{itemize}
        \item[] Ik kan mij soms te veel verdiepen in een onderwerp. Dit neemt vaak veel tijd in beslag en die heb ik eigenlijk niet.
        \item[] Uit eerdere ervaringen is gebleken dat ik door het thuiswerken zeer ongestructureerd kan werken.
    \end{itemize}

    Op het gebied van het project:
    \begin{itemize}
        \item[] Wanneer een project of een bepaalde zaak niet duidelijk is omschreven en ik hier vervolgens verder onderzoek naar doe/na vraag over doe en dit niets opheldert, irriteert mij dit ontzettend.
    \end{itemize}

    Op het gebied van teamverband:
    \begin{itemize}
        \item[] Ook kan ik heel erg vasthouden aan standpunten die ik heb. Ik ben moeilijk te overtuigen als de tegenpartij geen goede argumenten geeft .
        \item[] Ik kan me ontzettend storen aan proftaak groepjes met 'onge\"{\i}nteresseerde' individuen.
    \end{itemize}


    \section{Kansen}\label{sec:kansen}
    In dit hoofdstuk ga ik uitleggen waar kansen voor mij liggen binnen mijn project.

    \begin{itemize}
        \item[] Het toepassen van documentatie op een professionele manier die nuttig wordt geacht.
        \item[] Het team motiveren en aansporen om hun taken tot een goed einde brengen en dat ik mij hierbij ga proberen open te stellen voor andermans bevindingen.
        \item[] Omdat ik vorig semster ongestructureerd te werk ben gegaan wil ik het dit semester anders gaan aanpakken, door middel van een strak schema.
    \end{itemize}

    \section{Bedreigingen}\label{sec:bedreigingen}
    In dit hoofdstuk ga ik uitleggen wat eventuele bedreigingen kunnen zijn tijdens het ontwikkelen van mijn individueel project.
    In het verleden ben ik tegen een aantal dingen aangelopen die ik nu als een bedreiging kan zien voor het ontwikkelen van het project deze zijn als volgt:
    \begin{itemize}
        \item[] Kennis te laat opdoen die nodig is om mijn leerdoelen aan te tonen.
        \item[] Wanneer het proftaak-project meer tijd vraagt dan ik verwacht had en ik mij daarop moet verleggen.
        \item[] Het (online) werken in teamverband en de communicatie is door corona soms nog lastig.
        \item[] Wanneer ik te gefocust ben op een bepaald probleem verlies ik het zicht over het hele plaatje.
        \item[] Door mijn dyslexie vind ik het documenteren en daarmee het schrijven van onderzoeken ontzettend lastig en tijdrovend.
    \end{itemize}
    Ik ga proberen de dingen die hierboven zijn omschreven te verbeteren in dit semester zodat ik het beste eruit kan halen.

\end{document}
